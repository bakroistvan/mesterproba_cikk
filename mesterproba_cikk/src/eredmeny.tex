 \section{Eredmények} 
	\subsection{MATLAB implementációk}
	A referenciaként szolgáló MATLAB algoritmus lineáris programszervezést
	alkalmazva az elérhető fajlagos futási idő $\sim100 ms$.
	
	A kód minimális változtatásával elérhető a párhuzamos végrehajtás. Ezt a
	\texttt{for} ciklusok Parallel Toolbox beli \texttt{parfor} utasítására
	cserélve érhetjük el. 4 processzormaggal rendelkező PC esetén ilyenkor
	közel a negyedére csökken a futási idő.
	
	\subsection{OpenCL implementációk}
	OpenCL keretrendszer segítségével írt programot a GPU-n futtatva a
	\ref{table:openresult} táblázatban látható eredményeket kapjuk.
	Csupán a globális memóriát használva a referenciához képest romlik a
	teljesítmény. Ezt a videókártya prediktív cache nélküli kialakításának és a
	globális memórája okozta kiéheztetésnek tudhatjuk be.
	A lokális memória használata a futási időt drasztikusan le tudja
	csökkenteni, ami a korábban ismertetett memória szervezési gondolatok
	helyességét igazolja.
	 
	\begin{table}[!t]
		\renewcommand{\arraystretch}{1.5}
		% if using array.sty, it might be a good idea to tweak the value of
		% \extrarowheight as needed to properly center the text within the cells
		\caption{OpenCL futási idő eredmények $12\times12$ mérési pontra}
		\label{table:openresult}
		\centering
		% Some packages, such as MDW tools, offer better commands for making tables
		% than the plain LaTeX2e tabular which is used here.
		\resizebox{\columnwidth}{!}{
		\begin{tabular}{l|r|r|r}
			 & Globális memória & Lokális memória, ha befér & Lokális memória bufferelés\\ \hline
			\parbox{2.5cm}{Globális tranzakciók száma átlagosan} & 12 \times 12\times
			32.3 & 12 \times 12 \times 32.3 & 12 \times 12 \times 32.3\\
			\parbox{2.5cm}{Lokális tranzakciók száma átlagosan} & 0 &
			0.48 \times 12 \times 12 \times 30 & 2.08 \times 12 \times12 \times 32.3\\
			Futási idő & 5990 ms & 2530 ms & 510 ms\\
			Fajlagos futási idő & 410 ms & 170 ms & 3.5 ms 
		\end{tabular}
		}
	\end{table}
	
