\section{Összegzés}
	A cikkben összefoglaltam az AFM felületi töltéssűrűség méréstechnikáját, amiben azonosítottam a
	kapacitás értékének kritikus voltát. Ezidáig a kapacitás értékének a \cite{Hudlet1998,Butt20051}
	szerinti közelítéseket tartalmazó analitikus eredményt lehetett felhasználni.
	
	Feladatként ezen kapacitás értékét számító szimulátor építését tűztem ki, aminek elfogaható időn
	belül kell eredményt szolgáltatnia. A párhuzamosítás lehetősége triviálisan adódott.
	A hordozhatóság és a gyorsabb végrehajtás végett a szimulátort OpenCL 
	környezetben implementáltam, ami a szimulátor heterogén multiproszesszoros környezetben való
	futtatását lehetővé teszi.
	
	Az eredményeket ismertetve a szimulátor futási idejében látványos gyorsulást tapasztaltam, ami
	az érdesebb felületek esetén is elfogadhatóan pontos eredményt tud szolgáltatni. 
	
	A szimuláció felhasználásával történő töltéssűrűség származtatása még várat magára, ugyanígy ezen
	származtatás validálására való mérési összeállítás kidolgolgozása.
	A szimulátor magját képező lineáris egyenletrendszer iterációs megoldó
	konvergenciájának bizonyítása és az alternatív direkt megoldó vizsgálata is további feladat.
